\newpage
\section{Diskussion}
\label{sec:diskussion}
Werden die geometrischen Daten aus Tab. \ref{tab:rohrdaten} von WÜ 1 mit \mbox{WÜ 4} und \mbox{WÜ 6} verglichen so lässt sich feststellen, dass sich \mbox{WÜ 1} und \mbox{WÜ 4} im Durchmesser und \mbox{WÜ 1} und \mbox{WÜ 6} lediglich in der Länge des Rohres unterscheiden. \linebreak
Aufgrund der Tatsache, dass \mbox{WÜ 6} die vergleichsweise schlechteste Wärmeübertragung aufweist, lässt sich die Vermutung aufstellen, dass die Rohrlänge einen deutlich höheren Einfluss auf die Effizienz der Rohrwärmetauscher hat, als der Durchmesser. \linebreak 
Zudem ergibt sich, dass im Vergleich von \mbox{WÜ 4} und \mbox{WÜ 6} ein größerer Rohrquerschnitt eine kürzere Rohrlänge unter Umständen kompensieren könnte.\linebreak 
Als eindeutige Erkenntnisse aus diesem Versuch gehen hervor, dass vorrangig der luftseitige Wärmeübergang den Charakter der Wärmeübertragung bestimmt und das WÜ 1 am nächsten an dem idealen Prozess dafür liegt. \\
Um die beschriebenen Vermutungen bezüglich Rohrlänge und Rohrquerschnitt  verifizieren zu können, sollten weitere Untersuchungen durchgeführt werden.