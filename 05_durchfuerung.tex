\newpage
\section{Durchführung}
\label{sec:durchfuerung}

Der Versuch der Flammpunktprüfung erfolgte in diesem Fall nach der \linebreak Pensky-Martens-Methode in einem geschlossenen Tiegel nach der Norm DIN EN ISO 2719.\linebreak
Hierfür wurden die Proben jeweils in den sauberen Tiegel der automatische Messapperatur gefüllt und ein Wert für den zu erwartenden Flammpunkt eingegeben. Das Messgerät beginnt infolgedessen die etappenweise Prüfung des Flammpunkts mittels Glühdrahts unter konstanter Temperaturerhöhung. Ist der Flammpunkt um thermische Sensoren detektiert worden, aufgrund eines kurzen Temperaturmaximums durch die Entflammung, wird die Messung beendet.\\
Für die Prüfung der Weinprobe auf ihren Volumenanteil mittel Flammpunktprüfung wird eine 
Rein-Ethanol-Wasser-Mischung mit der genau dem angegebenen an Ethanol (\SI{12}{\volp}) hergestellt und diese miteinander verglichen.
