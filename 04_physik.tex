\section{Theoretische Grundlagen}
\label{sec:theorie}

Grundlage für den Versuch stellte die Wärmeübertragung am Rohr da. So lässt sich der übertragende Wärmestrom über die spezifische Wärmekapazität, dem Massenstrom, sowie aus der Differenz zwischen eingehender und ausgehender Temperatur des Stromes.
\begin{flalign}
 	\dot{Q} &= \dot{m}*c_p*\Delta T\\
 	\dot{Q} &= \dot{m}*c_p*(T_\omega-T_\alpha)
\end{flalign}

\begin{flalign}
	\Delta T_{_{\ln}}	&=  \frac{\Delta T_A-\Delta T_B}{\ln\left(\frac{\Delta T_A}{\Delta T_B}\right)}\\[1mm]
	\dot{Q} 	 &= U_a *A*\Delta T_{_{\ln}}\\
	U_a				&= \frac{\dot{Q}}{\Delta T_{_{\ln}}}
\end{flalign}

\begin{flalign}
	U_a		&= \left(\frac{1}{\alpha_i}+\sum\frac{d}{\lambda}+\frac{1}{\alpha_a}\right)^{-1}\\
	\alpha_i &=\left(\frac{1}{U_a}-\sum\frac{d}{\lambda}+\frac{1}{\alpha_a}\right)^{-1}
\end{flalign}

\begin{flalign}
	d_H \, (\text{Rohr})	&= D_i-d_a
\end{flalign}

\begin{flalign}
	Re	&= \frac{d*w}{\nu}
\end{flalign}

\begin{flalign}
		Pr	&= \frac{c_p*\nu*\rho}{\lambda}
\end{flalign}

\begin{flalign}
	Nu_{\text{ideal}}	&= 0,023*\left(Re^2*Pr\right)^{0,4}
\end{flalign}
\begin{flalign}
	Nu 	&= \frac{\alpha*d}{\lambda}\\[1mm]
	\alpha	&=  \frac{Nu*\lambda}{d}
\end{flalign}

\begin{flalign}
	a	&= e^{\ln(Nu)}
\end{flalign}
\begin{flalign}
	b &= f'\left(\ln(Re^2*Pr),\ln(Nu)\right)
\end{flalign}

\textcolor{red}{Stoff und Rohrdaten}
