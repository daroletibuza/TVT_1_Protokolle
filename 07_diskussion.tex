\newpage
\section{Fazit zur Technischen Handhabung}
\label{sec:diskussion}
Drei der vier Proben wurden mittels der Flammpunktprüfung als Gefahrstoffe der Klasse R10/H226 eingeordnet. Bei den Ethanol-Proben und dem alkoholischen Getränk sollte daher für Lagerung, Transport und Verarbeitung darauf geachtet werden, dass falls eine Entzündung oder ein Aufflammen der Proben unerwünscht ist, dass diese Stoffe gekühlt werden bzw. die Raum Temperatur unterhalb des Gefahrenpotentials bleiben sollte. Gerade wenn die die Umgebungstemperatur von den Jahreszeiten abhängt, sollten entsprechende Kühlsysteme für große Mengen beispielsweise im Sommer zur Verfügung stehen. Auch die Vermeidung von Zündquellen in der Nähe dieser Stoffe ist daher zu empfehlen. Der Dieselprobe ist nach der Norm DIN EN ISO 2719 keiner Gefahrenklasse zuzuordnen. Sie ist nach dieser Methodik lediglich als eine Flüssigkeit mit einem Flammpunkt über \SI{55}{\celsius} einzuschätzen. 