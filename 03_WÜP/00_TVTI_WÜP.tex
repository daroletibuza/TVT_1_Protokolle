%Dokumentklasse

%draft als optionohne bilder für bessere performance
%\documentclass[a4paper,12pt,]{scrreprt}

%normal mit Bildern
\documentclass[
a4paper,
11pt,
draft=True]
{scrartcl}

%Section als Chapter
\RedeclareSectionCommand[%
%beforeskip = -1sp plus -1sp minus -1sp,% kleinster negativer Wert, um den Absatzeinzug nach der Überschrift zu verhindern.
afterskip = 1.5 \baselineskip plus -1sp minus 1sp,
font = \Huge,
]{section}

\usepackage[left= 3cm,right = 3cm, bottom = 3cm,top = 3cm]{geometry}
%\usepackage[onehalfspacing]{setspace}

% ============= Packages =============
% Dokumentinformationen
\usepackage[
pdftitle={Praktikum - Umwelttechnik},
pdfsubject={},
pdfauthor={Roman-Luca Zank},
pdfkeywords={},	
%Links nicht einrahmen
hidelinks
]{hyperref}

%nur Text zum prüfen des Umfangs

% Standard Packages
%\usepackage[bottom]{footmisc}
\usepackage[utf8]{inputenc}
\usepackage[ngerman]{babel}

\usepackage[T1]{fontenc}
%\usepackage{helvet}
\usepackage{rotating}

%\renewcommand{\familydefault}{\sfdefault}

\usepackage{graphicx}
\graphicspath{{img/}}
\usepackage[table]{xcolor}
\setlength\arrayrulewidth{0.8pt}
\usepackage{mhchem}
\usepackage{fancyhdr}
\usepackage{lmodern}
\usepackage{color}
\usepackage[bottom]{footmisc}
\usepackage{setspace}\usepackage{threeparttable}
%==================================================================
%\begin{threeparttable} 
%	\begin{tabular}{|l|c|r|} 
%		\hline 
%		A & B & C \\
%		\hline
%		1 & 2 & 3 \tnote{1} \\
%		\hline
%	\end{tabular} 
%	\begin{tablenotes}\footnotesize 
%		\item[1] Prognose 2003 
%	\end{tablenotes}
%====================================================================
\usepackage{placeins}
\usepackage{booktabs}
\usepackage{caption}
\usepackage[list=true]{subcaption}
\usepackage[longtable]{multirow}
\usepackage{tikz}
\usepackage{pgfplots}
\pgfplotsset{/pgf/number format/use comma}
\usepackage{lastpage}
%\usepackage{ulem}
\usepackage{mathtools}
\usepackage{adjustbox}
\usetikzlibrary{patterns}
\usepackage{pdfpages}
\usepackage{colortbl}

%Einheitenpackage
\usepackage{siunitx}  
\sisetup{	locale = DE, 
	per-mode=fraction,
	inter-unit-product=\ensuremath{\cdot},
	detect-weight = true,
	quotient-mode=fraction
}
%neue Einheiten definieren
\DeclareSIUnit\xyz{xyz}	
\DeclareSIUnit\rpm{rpm}	
\DeclareSIUnit\mws{mWS}	
\DeclareSIUnit\degrees{^\circ}	
\DeclareSIUnit\volp{V\%}
\DeclareSIUnit\mpp{m\%}
\DeclareSIUnit\mmWS{mmWS}
\DeclareSIUnit\cmt{\raiseto{3}\meter}
\DeclareSIUnit\sqm{\raiseto{2}\meter}

%Automatisch cdot statt *
\DeclareMathSymbol{*}{\mathbin}{symbols}{"01}


%Tabelle
\usepackage{tabularx}
\usepackage{tabulary}

%nur letzte Zeile der Gleichung nummerieren
\makeatletter
\def\Let@{\def\\{\notag\math@cr}}
\makeatother

%angepasster \today Command
\newcommand{\leadingzero}[1]{\ifnum #1<10 0\the#1\else\the#1\fi}
\newcommand{\todayDE}{\leadingzero{\day}.\leadingzero{\month}.\the\year}

% zusätzliche Schriftzeichen der American Mathematical Society
\usepackage{amsfonts}
\usepackage{amsmath}

%Abkürzungsverzeichnis
\usepackage{acronym}

%kein Abstand bei neuem Kapitel vom Seitenanfang
%\vspace*{2.3\baselineskip} = ORIGINAL
%\renewcommand*{\chapterheadstartvskip}{\vspace*{.0\baselineskip}}

%nicht einrücken nach Absatz
\setlength{\parindent}{0pt}

\urlstyle{same}


% ============= Kopf- und Fußzeile =============
\pagestyle{fancy}
%
\lhead{}
\chead{}
\rhead{}%\slshape }%\leftmark}
%%
\lfoot{}
\cfoot{}
\rfoot[{\thepage\ of \pageref*{LastPage}}]{Seite \thepage\ von \pageref*{LastPage}}
%%
\renewcommand{\headrulewidth}{0pt}
\renewcommand{\footrulewidth}{0pt}
%\renewcommand{\chapterpagestyle}{fancy}

%Fußnotelinie
%\let\footnoterule

%Fußnote mit Klammer
\renewcommand*{\thefootnote}{(\arabic{footnote})}

%Abb. statt Abbildung
\addto\captionsngerman{%
	\renewcommand{\figurename}{Abb.}%
	\renewcommand{\tablename}{Tab.}%
}

% ============= Package Einstellungen & Sonstiges ============= 
%Besondere Trennungen
%\hyphenation{De-zi-mal-tren-nung}
\usepackage[none]{hyphenat}
\hyphenpenalty=5000
\tolerance=5000
\providecommand\phantomsection{}

\usepackage{mathtools}


% ============= Dokumentbeginn =============

\begin{document}
%Seiten ohne Kopf- und Fußzeile sowie Seitenzahl
\pagestyle{empty}

%\begin{center}
\begin{tabular}{p{\textwidth}}


\begin{center}
\includegraphics[scale=0.75]{logos.jpg}\\
\end{center}


\\

\begin{center}
\LARGE{\textsc{
Protokoll \\
Thermische Verfahrenstechnik I\\
}}
\end{center}

\\

%\begin{center}
%\large{Fakultät für Muster und Beispiele \\
%der Hochschule Musterhausen \\}
%\end{center}
%
%\\
\begin{center}
	\textbf{\Large{Bestimmung des Wärmeübergangskoeffizienten}}
\end{center}

\\ \\
%\begin{center}
%zur Erlangung des akademischen Grades\\
%Bachelor of Engineering
%\end{center}


%\begin{center}
%vorgelegt von
%\end{center}

\begin{center}
\Large{\textbf{Teilnehmer:}} \\ 
\end{center}
\begin{center}
\large{Willy Messerschmidt \\
	Roman-Luca Zank} \\
\end{center}

\\ \\ \\

\begin{center}
\begin{tabular}{lll}
\large{\textbf{Gruppe:}} & & \large{J}\\
&&\\
\large{\textbf{Protokollführer:}} & & \large{Roman-Luca Zank}\\
&&\\
\large{\textbf{Datum der Versuchsdurchführung:}}&& \large{Online}\\
&&\\
\large{\textbf{Abgabedatum:}}&& \large{\today}
\end{tabular}
\end{center}

\\ \\ \\ \\ \\ \\ \\ \\ \\
\large{Merseburg den \today}

\end{tabular}
\end{center}


%\include{14_danksagungen}

%\include{15_zusammenfassung}

% Beendet eine Seite und erzwingt auf den nachfolgenden Seiten die Ausgabe aller Gleitobjekte (z.B. Abbildungen), die bislang definiert, aber noch nicht ausgegeben wurden. Dieser Befehl fügt, falls nötig, eine leere Seite ein, sodaß die nächste Seite nach den Gleitobjekten eine ungerade Seitennummer hat. 
\cleardoubleoddpage

% Pagestyle für Titelblatt leer
\pagestyle{empty}

%Seite zählen ab
\setcounter{page}{0}

%Titelblatt
\begin{center}
\begin{tabular}{p{\textwidth}}


\begin{center}
\includegraphics[scale=0.75]{logos.jpg}\\
\end{center}


\\

\begin{center}
\LARGE{\textsc{
Protokoll \\
Thermische Verfahrenstechnik I\\
}}
\end{center}

\\

%\begin{center}
%\large{Fakultät für Muster und Beispiele \\
%der Hochschule Musterhausen \\}
%\end{center}
%
%\\
\begin{center}
	\textbf{\Large{Bestimmung des Wärmeübergangskoeffizienten}}
\end{center}

\\ \\
%\begin{center}
%zur Erlangung des akademischen Grades\\
%Bachelor of Engineering
%\end{center}


%\begin{center}
%vorgelegt von
%\end{center}

\begin{center}
\Large{\textbf{Teilnehmer:}} \\ 
\end{center}
\begin{center}
\large{Willy Messerschmidt \\
	Roman-Luca Zank} \\
\end{center}

\\ \\ \\

\begin{center}
\begin{tabular}{lll}
\large{\textbf{Gruppe:}} & & \large{J}\\
&&\\
\large{\textbf{Protokollführer:}} & & \large{Roman-Luca Zank}\\
&&\\
\large{\textbf{Datum der Versuchsdurchführung:}}&& \large{Online}\\
&&\\
\large{\textbf{Abgabedatum:}}&& \large{\today}
\end{tabular}
\end{center}

\\ \\ \\ \\ \\ \\ \\ \\ \\
\large{Merseburg den \today}

\end{tabular}
\end{center}
 %Prokolle
%\include{01_titel2} %Seminar-/Abschlussarbeit

% Pagestyle für Rest des Dokuments
\pagestyle{fancy}

%Inhaltsverzeichnis
\tableofcontents
\thispagestyle{empty}

%Inhalt
%
%Verzeichnis aller Bilder
\label{sec:bilder}
\listoffigures
\addcontentsline{toc}{chapter}{Abbildungsverzeichnis}
\thispagestyle{empty}

%Verzeichnis aller Tabellen
\label{sec:tabellen}
\listoftables
\addcontentsline{toc}{chapter}{Tabellenverzeichnis}
\thispagestyle{empty}



%%Abkürzungsverzeichnis
%\setlength{\columnsep}{20pt}
%\twocolumn
%\addchap{Nomenklatur}
%\label{sec:abkurzung}
%\begin{acronym}
%\acro{kf}[$\text{k}_\text{f}$]{Durchlässigkeitsbeiwert}
%\acro{t}{Durchlaufzeit}
%\acro{tm}[$\text{t}_\text{m}$]{Mittlere Durchlaufzeit}
%\acro{V}{Volumen}
%\acro{h}{Höhe der Wassersäule}
%\acro{Q}{Volumenstrom}
%\acro{l}{Durchströmte Länge}
%\acro{A}{Grundfläche}
%\acro{d}{Durchmesser}
%
%\end{acronym}
%\subsubsection{Aufrufen einer Abkürzung}
%\acs{rT}
%\begin{verbatim}
%\acs{Abkürzung}
%\end{verbatim}

%\includepdf[]{Deckblatt}
\pagebreak
\section{Einleitung}
\label{sec:einleitung}
Im folgenden Protokoll werden generierte Messdaten zum Versuch \textit{WÜK} ausgewertet. Ziel ist es mit Hilfe der erklärenden Videos zum Praktikum und Mittels der gegebenen Messdaten den Wärmeübergangskoeffizient $\alpha_L$ für turbulente Luftströmungen zu bestimmen. Dafür werden  drei verschiedene Rohre unter unterschiedlichen Volumenströmen der Luft untersucht. Die Wärmeübertragung mit Wasser erfolgt  in diesem Versuch mittels Gleichstrom.
Darüber hinaus sind, mittels Nusseltzahl $Nu$ und der Nusseltparameter $a$ und $b$, Bewertungen zur Wärmeübertragung der verschiedenen Rohre und Volumenströme abzugeben.







\section{Theoretische Grundlagen}
\label{sec:theorie}

Grundlage für den Versuch stellte die Wärmeübertragung am Rohr dar. So lässt sich der übertragene Wärmestrom über die spezifische Wärmekapazität, dem Massenstrom, sowie aus der Differenz zwischen eingehender und ausgehender Temperatur des Stromes berechnen.
\begin{flalign}
 	\dot{Q} &= \dot{m}*c_p*\Delta T\\
 	\dot{Q} &= \dot{m}*c_p*(T_\omega-T_\alpha)\\
 	 \dot{Q} &= \dot{V}*\rho*c_p*(T_\omega-T_\alpha)
\end{flalign}
Da die generierten Messwerte dennoch ähnlich real gemessenen Werten sind, wird für die Auswertung ein Korrekturterm für den Volumenstrom eingeführt. Dieser hält die Abweichungen im Volumenstrom fest und korrigiert den Wert für den abgegebenen bzw. aufgenommenen Wärmestrom. Mit Hilfe des Korrekturwertes entspricht die abgegebene Wärmemenge $\dot{Q}_{ab}$ der aufgenommenen Wärme $\dot{Q}_{auf}$.
\begin{flalign}
	\dot{Q} &= \left(\dot{V}\pm \Delta \dot{V}\right)*\rho*c_p*(T_\omega-T_\alpha)
\end{flalign}
Bestimmt wird der Korrekturvolumenstrom $\Delta \dot{V}$ durch gleichsetzen der angepassten Gleichungen für die Wärmeströme.
\begin{flalign}
	\dot{Q}_{ab} &= \dot{Q}_{auf} \\
	 \left(\dot{V}_{ab}+\Delta \dot{V}\right)*\rho*c_p*(T_{\omega,i}-T_{\alpha,i}) &= \left(\dot{V}_{auf}-\Delta \dot{V}\right)*\rho*c_p*(T_{\omega,j}-T_{\alpha,j})
\end{flalign}

Um das Verhältnis zwischen einzusetzender Pumpleistung aufgrund von Druckverlusten und dem übertragenen Wärmestrom berechnen zu können, ist die folgende Gleichung für die Pumpenleistung notwendig:
\vspace*{-3mm}
\begin{flalign}
	P_{Pumpe} &= \Delta p * \dot{V}
\end{flalign}

Unter der Annahme, dass eine elektrische Kreiselpumpe mit einem Wirkungsgrad von \SI{80}{\percent } eingesetzt wird, ergibt sich für die einzusetzend, elektrische Leistung:
\begin{flalign}
	P_{elektr.} &= \frac{P_{Pumpe}}{0,8}
\end{flalign}

Für die weitere Charakterisierung der Strömung außerhalb des Rohrs wird der hydraulische Rohrdurchmesser $d_H$ als theoretische Größe eingeführt. Er soll sicherstellen, dass die vorherrschenden, turbulenten Strömungen ändernd gut beschrieben werden können.
\begin{flalign}
	d_H \, (\text{Rohr})	&= D_i-d_a
\end{flalign}
Solche Größen sind unter anderem die Reynoldszahl $Re$ zur Charakterisierung der Strömung als turbulent, laminar oder einem Übergangszustand.
\begin{flalign}
	Re	&= \frac{d*w}{\nu}
\end{flalign}
Die Prandtl-Zahl hingegen gibt das Verhältnis zwischen kinematischer Viskosität $\nu$ und der Temperaturleitfähigkeit $a$ an.
\begin{flalign}
		Pr	&= \frac{c_p*\nu*\rho}{\lambda}
\end{flalign}
Idealer Weise lässt sich so aus den beiden zuletzt genannten Größe eine weitere, dimensionslose Kennzahl definieren um den konvektiven Wärmeübergang zwischen einer festen Oberfläche und einem strömenden Fluid zu beschreiben. Diese Kenngröße ist die \textsc{Nußelt}-Zahl $Nu$. Für die äußere Wasserströmung lässt sich dieser Übergang als ideal annehmen, da der Wärmeübergang der Luft deutlich stärker die Übertragung beeinflusst.
\begin{flalign}
	Nu_{\text{ideal}}	&= 0,023*\left(Re^2*Pr\right)^{0,4}
\end{flalign}
Aus einem weiteren Zusammenhang der \textsc{Nußelt}-Zahl gegenüber der Wärmekonvektion, lässt sich so der Wärmeübergangkoeffizient $\alpha_a$ für die wasserseitige Wärmeübertragung berechnen.
\begin{flalign}
	Nu 	&= \frac{\alpha*d}{\lambda}\\[1mm]
	\alpha	&=  \frac{Nu*\lambda}{d}
\end{flalign}

Aus den berechneten Wärmeübergangskoeffizienten, den geometrischen Daten der Wärmeübertrager, sowie den Wärmeleitkoeffizienten des Fluides lässt sich als weiterer Kennparameter der Wärmedurchgangswiderstand $U_a$ einführen. Dieser charakterisiert ebenfalls den Wärmeübergangsprozess.
\begin{flalign}
	U_a		&=\left(\frac{d_a}{\alpha_i*d_i}+\frac{d_a}{2*\lambda}*\ln\left[\frac{d_a}{d_i}\right]+\frac{1}{\alpha_a}\right)^{-1}
\end{flalign}

Im Folgenden sind die genutzten geometrischen Rohrdaten, sowie die Stoffdaten hinterlegt.
\vspace*{-7mm}
\begin{table}[h!]
	\centering
	\caption{Rohrdaten}
	\rowcolors{2}{white}{gray!25}
	\label{tab:rohrdaten}%
	\renewcommand{\arraystretch}{1.2}
	%\resizebox{14cm}{!}{
		\begin{tabulary}{1.15\textwidth}{C|C|C}
			\hline
			&\textbf{warmseitig} (Innenrohr)& \textbf{kaltseitig} (Doppelmantel) \\
			\hline
			$d_i \left[\si{\milli \meter}\right]$&10&16\\
			$d_a \left[\si{\milli \meter}\right]$&13&- \\
			\hline
			$L \left[\si{ \meter}\right]$&\multicolumn{2}{c}{7,5}\\
			\hline
	\end{tabulary}
%}
\end{table}%
\FloatBarrier
\begin{table}[h!]
	\centering
	\caption{Stoffdaten}
	\rowcolors{2}{white}{gray!25}
	\label{tab:stoffdaten}%
	\renewcommand{\arraystretch}{1.3}
	\resizebox{15cm}{!}{
	\begin{tabulary}{1.3\textwidth}{L|L|L}
		\hline
		\textbf{Stoffwert }&\textbf{Einheit}& \textbf{Gleichung} \\
		\hline
		Wärmeleitfähigkeit (Stahl)& $\left[\si{\watt\per \meter\per \kelvin }\right]$&$\lambda_{Stahl}=15$\\
		Dichte (Wasser)&$\left[\si{\kg\per\cmt }\right]$&$\rho=1005,7-0,375*T$ \\
		Wärmeleitfähigkeit (Wasser)& $\left[\si{\kilo \joule \per \meter\per \kelvin\per \hour }\right]$&$\lambda_{Wasser}=2,0107+0,007606*T-0,000033467*T^2$\\
		kinematische Viskosität (Wasser) & $\left[\si{\sqm\per \second}\right]$&$\ln\left(\nu\right) = -13,2883-0,0280596*T+0,000112275*T^2$\\
		spezifische Wärmekapazität (Wasser) & $\left[\si{\kilo \joule \per \kg \per \kelvin }\right]$ & $c_p = 4,185$\\
		\hline
	\end{tabulary}
	}
\end{table}%
\FloatBarrier



%\newpage
\section{Geräte und Chemikalien}
\label{sec:geraete}

\textbf{Geräte:}
\begin{itemize}
\item Trockenofen Mettler-Toledo DO-302
\item Waage
\item Titrator T50, \textsc{Karl-Fischer}-Titrator (Fa. Mettler)
\item Computer mit Software \textsc{LabX}
\item Mikroliterspritze
\end{itemize}

\vspace*{5mm}

\textbf{Proben/Chemikalien:}
\begin{itemize}
\item \textsc{Karl-Fischer}-Reagenz (Einkomponentig)
\item deionisiertes Wasser
\item Isopropanol (2-Propanol)
\item Polyamid

\end{itemize}







\newpage
\section{Durchführung}
\label{sec:durchfuerung}

Der Versuch der Flammpunktprüfung erfolgte in diesem Fall nach der \linebreak Pensky-Martens-Methode in einem geschlossenen Tiegel nach der Norm DIN EN ISO 2719.\linebreak
Hierfür wurden die Proben jeweils in den sauberen Tiegel der automatische Messapperatur gefüllt und ein Wert für den zu erwartenden Flammpunkt eingegeben. Das Messgerät beginnt infolgedessen die etappenweise Prüfung des Flammpunkts mittels Glühdrahts unter konstanter Temperaturerhöhung. Ist der Flammpunkt um thermische Sensoren detektiert worden, aufgrund eines kurzen Temperaturmaximums durch die Entflammung, wird die Messung beendet.\\
Für die Prüfung der Weinprobe auf ihren Volumenanteil mittel Flammpunktprüfung wird eine 
Rein-Ethanol-Wasser-Mischung mit der genau dem angegebenen an Ethanol (\SI{12}{\volp}) hergestellt und diese miteinander verglichen.


\section{Ergebnisse und Berechnungen}
\label{sec:ergebnisse}
%Tabelle START

%Tabelle START
\vspace*{-2.5mm}
\renewcommand{\arraystretch}{1.2}
\begin{table}[h!]
	\centering
	\caption{Messergebnisse der untersuchten Proben}
	\label{tab:Messergebnisse der untersuchten Proben}
	\resizebox{15cm}{!}{
	\begin{tabulary}{1.1\textwidth}{C|C|C|L}
%		\hline
%		\multicolumn{3}{|c|}{Automatisch }\\
		\hline
		\textbf{Stoff}& \textbf{Flammpunkt}& \textbf{Gefahrenklasse} & \textbf{Begründung}\\
		\hline
		Wein & \SI{45}{\celsius} & R10/H226 & Flammpunkt zwischen 21 bis \SI{25}{\celsius} - entzündliche Flüssigkeit\\
		Unbekanntes Ethanol & \SI{45}{\celsius} & R10/H226 & Flammpunkt zwischen 21 bis \SI{25}{\celsius} - entzündliche Flüssigkeit\\
		Diesel & \SI{61}{\celsius} & - & Flüssigkeit mit Flammpunkt über \SI{55}{\celsius}\\
		Wasser-Ethanol- Gemisch (\SI{12}{\volp}) & \SI{47}{\celsius} & R10/H226 & Flammpunkt zwischen 21 bis \SI{25}{\celsius} - entzündliche Flüssigkeit\\
		\hline
	\end{tabulary}
	}
\end{table}
\FloatBarrier 
\vspace*{-4mm}
%Tabelle ENDE
%Start
\begin{figure}[h!]
	\centering
	\includegraphics[width=0.75\textwidth]{img/Diagramm}
	\caption{Flammpunkt in Abhängigkeit vom Ethanolanteil}
	\label{fig:dia}
\end{figure}
\FloatBarrier
%Ende

\subsection*{Bestimmung von Gewichts- und Volumenanteil der unbekannten Ethanolprobe}

Der in der Abbildung \ref{fig:dia} eingetragene Flammpunkt von \SI{42}{\celsius} lässt, mit der blau eingetragenen Abhängigkeit, einen Massenanteil der unbekannten Ethanol-Probe von \SI{12}{\volp} bestimmen.
\begin{flalign}
m_{\ce{Et}} 	&= m*\chi_{\ce{Et}}\\
&= \SI{1}{\kg}*\SI{12}{\percent}\\
&=\underline{\SI{0,12}{\kg}}
\end{flalign}
\begin{flalign}
m_{\ce{H2O}} 	&= m*\chi_{\ce{H2O}}\\
&= \SI{1}{\kg}*\SI{88}{\percent}\\
&= \underline{\SI{0,88}{\kg}}
\end{flalign}
\begin{flalign}
V_{\ce{Et}}		&= \frac{m_{\ce{Et}}}{\rho_{\ce{Et}}}\\
&= \frac{\SI{0,10}{\kg}}{\SI{0,789}{\kg\per\liter}}\\
&= \underline{\SI{0,152}{\liter}}
\end{flalign}
\begin{flalign}
V_{\ce{H2O}}	&= \frac{m_{\ce{H2O}}}{\rho_{\ce{H2O}}}\\
&= \frac{\SI{0,88}{\kg}}{\SI{1,000}{\kg\per\liter}}\\
&= \underline{\SI{0,880}{\liter}}
\end{flalign}
\begin{flalign}
V\%_{\ce{Et}}	&= \frac{V_{\ce{Et}}}{V_{\ce{Et}}+V_{\ce{H2O}}}\\
&= \frac{\SI{0,152}{\liter}}{\SI{0,152}{\liter}+\SI{0,880}{\liter}}\\
&\approx \underline{\SI{15}{\percent}}
\end{flalign} 

Die Umrechnung in Volumenprozent zeigt, dass sich für die gemessene unbekannte 
Ethanol-Probe ein Volumengehalt von \SI{15}{\volp} bestimmen lässt.
\subsection*{Vergleich der Weinprobe mit der Rein-Ethanol-Wasser-Mischung}
In Abbildung 1 ebenfalls eingetragen sind die Flammpunkte von je \SI{45}{\celsius} für das alkoholischen Getränk (Wein, Angabe \SI{12}{\volp}) und dem Rein-Ethanol-Wasser-Gemisch (angemischt \SI{12}{\volp}) mit einem Flammpunkt von \SI{47}{\celsius}. 
Aus Abbildung 1 ergeben sich daraus für das alkoholische Getränk \SI{12}{\mpp} Ethanol und für das Rein-Ethanol-Wasser-Gemisch \SI{10}{\volp}Ethanol.\\
\newpage
\textbf{Rückrechnung des Ethanol-Wasser-Gemisches von Massen- auf\linebreak Volumenprozent}
\begin{flalign}
m_{\ce{Et}} 	&= m*\chi_{\ce{Et}}\\
&= \SI{1}{\kg}*\SI{10}{\percent}\\
&=\underline{\SI{0,10}{\kg}}
\end{flalign}
\begin{flalign}
m_{\ce{H2O}} 	&= m*\chi_{\ce{H2O}}\\
&= \SI{1}{\kg}*\SI{90}{\percent}\\
&= \underline{\SI{0,90}{\kg}}
\end{flalign}
\begin{flalign}
V_{\ce{Et}}		&= \frac{m_{\ce{Et}}}{\rho_{\ce{Et}}}\\
&= \frac{\SI{0,10}{\kg}}{\SI{0,789}{\kg\per\liter}}\\
&= \underline{\SI{0,127}{\liter}}
\end{flalign}
\begin{flalign}
V_{\ce{H2O}}	&= \frac{m_{\ce{H2O}}}{\rho_{\ce{H2O}}}\\
&= \frac{\SI{0,90}{\kg}}{\SI{1,000}{\kg\per\liter}}\\
&= \underline{\SI{0,900}{\liter}}
\end{flalign}
\begin{flalign}
V\%_{\ce{Et}}	&= \frac{V_{\ce{Et}}}{V_{\ce{Et}}+V_{\ce{H2O}}}\\
&= \frac{\SI{0,127}{\liter}}{\SI{0,127}{\liter}+\SI{0,900}{\liter}}\\
&\approx \underline{\SI{12}{\percent}}
\end{flalign} 
Laut der Angabe auf der Weinverpackung mit \SI{12}{\volp} erwarten sich für das angesetzte 
Rein-Ethanol-Wasser-Gemisch und der alkoholischen Getränkeprobe dieselben Messwerte. Dies ist nicht der Fall. Das angesetzte Ethanol-Gemisch entspricht mit \SI{1}{\mpp} und den daraus berechneten \SI{12}{\volp} der Erwartung für die Flammprobe. Die Weinprobe ist jedoch mit \SI{12}{\mpp} und den sich daraus ergebenden \SI{15}{\volp} über dem Erwartungswert von \SI{12}{\volp}.\\
Grund für diese Abweichungen der alkoholischen Weinprobe könnten Fuselalkohole 
(z.B. Methanol, Propanol) sein, welche nicht dem Ethanol entsprechen, jedoch die Flammbarkeit des Weines selbst beeinflussen. So könnten diese Alkohole den Flammpunkt herabsetzen und damit eine verfälschte Bestimmung mit einem höheren Volumenanteil an Ethanol bewirken.


\newpage
\section{Diskussion}
\label{sec:diskussion}
Beginnend mit den Temperaturprofilen der Wärmetauscher für die Reihen- und Parallelschaltung werden an dieser Stelle die Messergebnisse diskutiert und bewertet. Die in der Abbildung \ref{dia:temp_profil} dargestellten Temperaturverläufe, welche über den Rohrabschnitt aufgetragen wurden,  geben erste Indizien zur Bewertung der beiden Fahrweisen. Es zeigt sich, dass die Temperaturverläufe für den parallelen Betrieb deutlich steiler zu laufen als für die Reihenschaltung. Zu erkennen ist ebenfalls, dass die Temperaturdifferenzen der Parallelschaltung größer sind, als die der Reihenschaltung. Somit lässt sich die Vermutung aufstellen, dass die Parallelschaltung die effizientere Wärmeübertragung für die jeweils eingestellten Volumina liefert. Jedoch sollte beachtet werden, dass für beide Verfahren unterschiedliche Volumenströme gefahren werden. Somit kann anhand dieses Diagramm lediglich die Fahrweisen mit den entsprechenden Betriebsparametern verglichen werden, geben jedoch keine Auskunft über den Vergleich von Reihen- und Parallelschaltung.\\

Anhand der Wärmeströme lassen sich nun beide Fahrweisen quantitativ in Form von Wärmeströme unterscheiden. Die Korrektur der Volumenströme ist hierbei notwendig und gibt Auskunft über den Fehler der Messungen am jeweiligen System. Auch an dieser Stelle der Auswertung wird deutlich, dass der Parallelbetrieb Vorteile in Bezug  wird deutlich, dass die Menge an übertragener Wärme höher ist als bei der Reihenschaltung. Es würde sich demnach auch hier wieder auf eine höhere Effizienz des Parallelbetriebs schließen lassen.\\

Zieht man in die Betrachtung nun auch die Wirtschaftlichkeit der jeweiligen Fahrweise mit ein, so setzt man die übertragene Wärmemenge in ein Verhältnis zur benötigen elektrischen Leistung für die Förderung des Fluides. Wirtschaftlichkeit bedeutet in diesem Fall, dass maximal viel Wärme übertragen wird, für eine minimale, elektrische Leistung der Pumpe. Demnach ist das Ziel möglichst hohe Werte für dieses Verhältnis zu erreichen. Es zeigt sich, dass die Werte für die Parallelschaltung deutlich wirtschaftlicher erscheinen als die Reihenschaltung, aufgrund der höheren Menge an übertragenen Wärme über den auszugleichenden Druckverlust. Auch unter diesem Aspekt besticht die Parallelschaltung als Fahrweise in diesem Versuchsaufbau.

\begin{figure}[h!]
	\begin{center}
		\resizebox{0.8\textwidth}{!}{
			\begin{tikzpicture}
			\begin{axis}[
			width  = 0.85*\textwidth,
			height = 7cm,
			major x tick style = transparent,
			ybar=2*\pgflinewidth,
			bar width=14pt,
			ymajorgrids = true,
			ylabel = {$U \, \left[\si{\watt \per \meter\per\kelvin}\right]$},
			symbolic x coords={WÜ1,WÜ2},
			xtick = data,
			scaled y ticks = false,
			enlarge x limits=0.25,
			ymin=0,
			legend cell align=left,
			legend style={
				at={(1,1.05)},
				anchor=south east,
				column sep=1ex
			}
			]
			%Reihenschaltung
			\addplot[style={black,fill=black,mark=none}]
			coordinates {(WÜ1, 31.13) (WÜ2,34.98)};
			
			%Paralaleschaltung
			\addplot[style={gray,fill=gray,mark=none}]
			coordinates {(WÜ1,36.13) (WÜ2,36.27)};
			
			
			\legend{Reihenschaltung, Parallelschaltung}
			\end{axis}
			\end{tikzpicture}
		}
		\caption{Grafischer Vergleich der Wärmedurchgangskoeffizienten}
		\label{dia:durchng}
	\end{center}
\end{figure}
\FloatBarrier



%\newpage
\section{Fehlerbetrachtung}
\label{sec:fehler}
Für die Fehlerbetrachtung ist davon auszugehen, dass jegliche Messeinrichtungen die Messwerte mit Fehlern in bestimmten Toleranzen aufnehmen. Weiterhin sind Vereinfachungen für die auswertenden Berechnungen angenommen worden, welche die ausgewerteten Ergebnisse ebenfalls verfälschen. Die vorliegenden Ergebnisse sind demnach nur eine Näherung an den realen Zustand. Dennoch sind sie ausreichend um die qualitativen Unterschiede zwischen Reihen- und Parallelschaltung aufzuzeigen.\\
Um einen Teil der fehlerbehafteten Größen für die einzelnen Wärmetauscher zu quantifizieren bzw. die Wärmeströme entsprechend zu korrigieren, wurde für diesen Versuch ein Korrekturvolumenstrom eingeführt. Für die jeweilige Schaltung und den jeweiligen Wärmetauscher sind diese Korrekturvolumenströme im Diagramm \ref{dia:korrrek} aufgetragen. Dabei ist zu erkennen, dass der Betrag des Korrekturvolumenstroms für den \mbox{Wärmeübertrager 1} am höchsten hervorsticht. Dies sollte bei der Bewertung der berechneten Daten berücksichtigt werden.\\
Weiterhin sind Vereinfachungen getroffen worden, dass sich beispielsweise die Volumenströme in der Parallelschaltung gleichmäßig aufteilen, sowie dass für bestimmte, temperaturabhängige Größen Näherungsgleichungen genutzt wurden.
\vspace*{-3mm}
\begin{figure}[h!]
	\begin{center}
		\resizebox{0.75\textwidth}{!}{
			\begin{tikzpicture}
			\begin{axis}[
			width  = 0.85*\textwidth,
			height = 7cm,
			major x tick style = transparent,
			ybar=2*\pgflinewidth,
			bar width=14pt,
			ymajorgrids = true,
			ylabel = {$\left|\Delta \dot{V}\right|\, \left[\si{\cmt\per \second}\right]$},
			symbolic x coords={WÜ1,WÜ2},
			xtick = data,
			%scaled y ticks = false,
			enlarge x limits=0.25,
			ymin=0,
			ymax=0.00001,
			legend cell align=left,
			legend style={
				at={(1,1.05)},
				anchor=south east,
				column sep=1ex
			}
			]
			%Reihenschaltung
			\addplot[style={black,fill=black,mark=none}]
			coordinates {(WÜ1, 9.7E-07) (WÜ2,1.71E-06)};
			
			%Paralaleschaltung
			\addplot[style={gray,fill=gray,mark=none}]
			coordinates {(WÜ1,8.00E-06) (WÜ2,1.64E-06)};
			
			
			\legend{Reihenschaltung, Parallelschaltung}
			\end{axis}
			\end{tikzpicture}
		}
		\caption{Grafischer Vergleich der Korrekturvolumenströme}
		\label{dia:korrrek}
	\end{center}
\end{figure}
\FloatBarrier

Auffallend für die Fehlerbetrachtung ist noch, dass die Messwerte für diesen theoretischen Versuch eine höhere Pumpleistung für die Reihen- als für Parallelleistung erfordert wird. Dies erscheint nicht als sinnvoll, da gerade durch die Aufteilung, die Lenkung und wieder Zusammenführung der Strömungen im Parallelbetrieb ein höherer Druckverlust entstehen würde als in der Reihenschaltung. Die Reihenschaltung erfordert nämlich keine Umlenkung oder Aufteilung des Fluidstroms.\\

Im Endeffekt lässt sich sagen, dass der Versuch erneut durchgeführt werden sollte, um die Ergebnisse dieses Versuches zu überprüfen. Weiterhin sollten beide Fahrweisen unter den selben Volumenströmen gefahren werden, da ein Vergleich sonst nur bedingt sinnvoll erscheint, wenn man lediglich die Schaltungen der Wärmetauscher überprüfen möchte.

%%\section*{Anhang}
\addcontentsline{toc}{section}{Anhang}
%\label{sec:anhang}
 
 
 
 

%%Praktikumsskript, Modul ………, Versuch …….., Prof. Musterprof. 
%DIN 12345, Jahr der Veröffentlichung 
%Link der Internetseite, Zugriffsdatum 
%Buchtitel, Autor, Verlag, Veröffentlichungsjahr 

%Literaturverzeichnis Bücher
\bibliography{Literatur}
\bibliographystyle{unsrtdin}
\addcontentsline{toc}{section}{Literaturverzeichnis}

%\include{09_erklaerung}

\end{document}
